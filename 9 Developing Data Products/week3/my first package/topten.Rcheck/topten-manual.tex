\nonstopmode{}
\documentclass[a4paper]{book}
\usepackage[times,inconsolata,hyper]{Rd}
\usepackage{makeidx}
\usepackage[utf8]{inputenc} % @SET ENCODING@
% \usepackage{graphicx} % @USE GRAPHICX@
\makeindex{}
\begin{document}
\chapter*{}
\begin{center}
{\textbf{\huge Package `topten'}}
\par\bigskip{\large \today}
\end{center}
\inputencoding{utf8}
\ifthenelse{\boolean{Rd@use@hyper}}{\hypersetup{pdftitle = {topten: Building a Prediction Model from Top 10 Features}}}{}\begin{description}
\raggedright{}
\item[Type]\AsIs{Package}
\item[Title]\AsIs{Building a Prediction Model from Top 10 Features}
\item[Version]\AsIs{1.0}
\item[Author]\AsIs{Antonio}
\item[Maintainer]\AsIs{Antonio }\email{yourself@somewhere.net}\AsIs{}
\item[Description]\AsIs{Funtions for building and predicting models from selecting the top 10 predictors in a dataset}
\item[License]\AsIs{GPL-3}
\item[Encoding]\AsIs{UTF-8}
\item[RoxygenNote]\AsIs{7.1.2}
\item[NeedsCompilation]\AsIs{no}
\end{description}
\Rdcontents{\R{} topics documented:}
\inputencoding{utf8}
\HeaderA{predict10}{Prediction with Top Ten features}{predict10}
%
\begin{Description}\relax
This function takes a set of coefficients produced by the \code{topten} function
and makes a prediction for each of the values provided in the input 'X' matrix.
\end{Description}
%
\begin{Usage}
\begin{verbatim}
predict10(X, b)
\end{verbatim}
\end{Usage}
%
\begin{Arguments}
\begin{ldescription}
\item[\code{X}] a n x 10 matrix containing n new observations

\item[\code{b}] is a vector of coefficients obtained from the \code{topten} function
\end{ldescription}
\end{Arguments}
%
\begin{Value}
a vector containing the predicted values
\end{Value}
\inputencoding{utf8}
\HeaderA{topten}{Building a model with top ten features}{topten}
%
\begin{Description}\relax
This function develops a prediction algorithm based on the top 10 features
in 'x' that are most predictive of 'y'
\end{Description}
%
\begin{Usage}
\begin{verbatim}
topten(x, y)
\end{verbatim}
\end{Usage}
%
\begin{Arguments}
\begin{ldescription}
\item[\code{x}] a n x p matrix of n observations and p predictors

\item[\code{y}] a vector of length n representing the response
\end{ldescription}
\end{Arguments}
%
\begin{Details}\relax
This function runs a univariate regression of y on each predictor in x and
calculates a p-value indicating the significance of the association, The final
set of 10 predictores is taken from the 10 smallest p-values
\end{Details}
%
\begin{Value}
a vector of coefficients from the final fitted model with top 10 features
\end{Value}
%
\begin{Author}\relax
Antonio Vitor
\end{Author}
%
\begin{SeeAlso}\relax
\code{lm}
\end{SeeAlso}
\printindex{}
\end{document}
